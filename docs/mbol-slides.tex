\documentclass{beamer}
\usepackage[normalem]{ulem}
\usepackage{graphicx}
\usepackage{verbatim}
\usepackage{amssymb}
\usepackage{xstring}
\usepackage{amsmath}
\newcounter{subjto}
\let\oldsum\sum
\renewcommand{\sum}{\oldsum\limits}
\newcommand{\maxf}[2]{\underset{#1}{\text{maximize}}&&#2}
\newcommand{\minf}[2]{\underset{#1}{\text{minimize}}&&#2}
\newcommand{\con}[2][]{
    \ifnum \value{subjto}>0
    \setcounter{subjto}{0}
    \\
    \text{subject to}
    \else
    ,
    \\
    \fi
    \IfSubStr{#2}{\mathbb{Z}}{
        &\hspace{0.3in}& \StrBefore{#2}{\in} & \in & \mathbb{Z}
        }{
        \IfSubStr{#2}{=}{
            &\hspace{0.3in}& \StrBefore{#2}{=} & = & \StrBehind{#2}{=}
            }{\IfSubStr{#2}{\le}{
                &\hspace{0.3in}& \StrBefore{#2}{\le} & \le & \StrBehind{#2}{\le}
                }{\IfSubStr{#2}{\ge}{
                    &\hspace{0.3in}& \StrBefore{#2}{\ge} & \ge & \StrBehind{#2}{\ge}
                    }{\IfSubStr{#2}{\in}{
                        &\hspace{0.3in}& \StrBefore{#2}{\in} & \in & \StrBehind{#2}{\in}
                        }{
                        error
                    }
                }
            }
        }
    }
    \IfStrEq{#1}{}{
        %        &\hspace{0.3in}& 
        } {
        ,&\hspace{0.3in}& \forall #1
    }
    
}
\newenvironment{mbol}
{
    \setcounter{subjto}{1}
    \begin{equation*}
        \begin{array}{ccrclcl}
        }
        {
            .\\
        \end{array}
    \end{equation*}
}

\begin{document}
\title{Math-Based Optimization Language (MBOL)}
\author{Andy Newell}
\frame{\titlepage}
\frame[t]{\frametitle{Optimization Program}
\begin{mbol}
\maxf{f}{\sum_{j \in V}(f_{s,j})}
\con{\sum_{ j \in V} (f_{i, j}) = \sum_{j \in V} (f_{j,i}), \; i \in V \setminus ( \{ s \} \cup \{ t \})}
\con{f_{i,j} \le w_{i,j}, \; i \in V, \; j \in V}
\con{f_{i,s} = 0, \;i \in V}
\con{f_{t,i} = 0, \;i \in V}
\end{mbol}

}
\frame[t]{\frametitle{Typical Optimization Experience}
\begin{enumerate}
\item Think about formulation (hours)
\item Write down problem in Latex source (minutes)
\item Compile Latex source for paper, presentations, or collaboration
\item Learn a solver's API and express problem with solver's API (hours)
\item Compile with standard compiler for use
\item Debug either problem (Steps 1 and 2) or translation to solver's API (Step 3)
\end{enumerate}
}
\frame[t]{\frametitle{Expressing Problem with Solver's API}
\begin{mbol}
\maxf{f}{\sum_{j \in V}(f_{s,j})}
\con{\sum_{ j \in V} (f_{i, j}) = \sum_{j \in V} (f_{j,i}), \; i \in V \setminus ( \{ s \} \cup \{ t \})}
\con{f_{i,j} \le w_{i,j}, \; i \in V, \; j \in V}
\con{f_{i,s} = 0, \;i \in V}
\con{f_{t,i} = 0, \;i \in V}
\end{mbol}

\begin{itemize}
\item $\sum_{j \in V}$: for loop over elements in some set $V$
\item $f_{i,j}$: variable of map indexed by two elements $i$ and $j$
\item $,\; i \in V$: for loop where each iteration creates a constraint
\item $\{s\} \cup \{t\}$: create set with elements $t$ and $s$
\end{itemize}
}
\frame[t]{\frametitle{Optimization Experience with MBOL}
\begin{enumerate}
\item Think about formulation (hours)
\item Write down problem in Latex source (minutes)
\item Compile Latex source for paper, presentations, or collaboration
\item \sout{Learn a solver's API and express problem with solver's API (hours)}
\item \sout{Compile with standard compiler for use}
\item \sout{Debug either problem (Steps 1 and 2) or translation to solver's API (Step 3)}
\item Compile with MBOL and standard compiler for use
\item Debug the problem (Steps 1 and 2)
\end{enumerate}
}
\frame[t]{\frametitle{How does MBOL Work?}
\begin{itemize}
\item Exploits common mathematics syntax that already avoids ambiguity
\begin{itemize}
\item Can learn types based on usage
\item Can dynamically construct necessary sets
\end{itemize}
\item Use common compiler tools (Lex/Yacc) to read and understand syntax
\item A lot of simple, automated translation to solver's API language
\end{itemize}
}
\frame[t]{\frametitle{MBOL Advantages}
\begin{itemize}
\item Rapid development and testing of optimization routines
\item Allow a much larger audience to use optimization solvers
\begin{itemize}
\item Many scientists cannot program
\item Those that can program, cannot do it well (we can barely do it ourselves)
\item Most scientists that use mathematics will know latex
\end{itemize}
\item Hands-n classroom education for optimization
\item Immediate understanding of unfamiliar source code
\begin{itemize}
\item Papers will include math-based versions of optimization as they are concise and understandable
\item Real source code of implementation is omitted for obvious reasons
\end{itemize}
\item Reproducability of experiments
\end{itemize}
}
\frame[t]{\frametitle{Making the Tool Publicly Available}
\begin{itemize}
\item License
\begin{itemize}
\item Chose MIT license, does nothing except prevents others from licensing your code for themselves
\item Interestingly, if I really cared about free software, interpretted language is better than compiler
\end{itemize}
\item Tool needs to be easily obtainable and usable
\begin{itemize}
\item Git-hub, free, cloud backups, version control
\begin{itemize}
\item git clone https://github.com/newell-purdue/mbol.git
\item cd mbol; make; sudo make install
\end{itemize}
\item In the future, Debian package would be nice
\begin{itemize}
\item sudo apt-get install mbol
\end{itemize}
\item Simple Makefile instead of heavy autotools
\begin{itemize}
\item More understandable
\item May miss obnoxious compatibility issues, may need this
\end{itemize}
\item Samples included to teach user, need good documentation
\end{itemize}
\end{itemize}
}
\frame[t]{\frametitle{Demo}
}
\end{document}
